\documentclass[english,a4paper,]{report}
\usepackage{lmodern}
\usepackage{amssymb,amsmath}
\usepackage{ifxetex,ifluatex}
\usepackage{fixltx2e} % provides \textsubscript
\ifnum 0\ifxetex 1\fi\ifluatex 1\fi=0 % if pdftex
  \usepackage[T1]{fontenc}
  \usepackage[utf8]{inputenc}
\else % if luatex or xelatex
  \ifxetex
    \usepackage{mathspec}
  \else
    \usepackage{fontspec}
  \fi
  \defaultfontfeatures{Ligatures=TeX,Scale=MatchLowercase}
\fi
% use upquote if available, for straight quotes in verbatim environments
\IfFileExists{upquote.sty}{\usepackage{upquote}}{}
% use microtype if available
\IfFileExists{microtype.sty}{%
\usepackage[]{microtype}
\UseMicrotypeSet[protrusion]{basicmath} % disable protrusion for tt fonts
}{}
\PassOptionsToPackage{hyphens}{url} % url is loaded by hyperref
\usepackage[unicode=true]{hyperref}
\hypersetup{
            pdftitle={Data Structures \& Algorithms},
            pdfauthor={Ricard Solé Casas},
            pdfborder={0 0 0},
            breaklinks=true}
\urlstyle{same}  % don't use monospace font for urls
\usepackage[margin=1in]{geometry}
\ifnum 0\ifxetex 1\fi\ifluatex 1\fi=0 % if pdftex
  \usepackage[shorthands=off,main=english]{babel}
\else
  \usepackage{polyglossia}
  \setmainlanguage[]{english}
\fi
\IfFileExists{parskip.sty}{%
\usepackage{parskip}
}{% else
\setlength{\parindent}{0pt}
\setlength{\parskip}{6pt plus 2pt minus 1pt}
}
\setlength{\emergencystretch}{3em}  % prevent overfull lines
\providecommand{\tightlist}{%
  \setlength{\itemsep}{0pt}\setlength{\parskip}{0pt}}
\setcounter{secnumdepth}{5}

% set default figure placement to htbp
\makeatletter
\def\fps@figure{htbp}
\makeatother

\usepackage{minted}
\usemintedstyle{autumn}

\usepackage{fontspec}
\setmonofont{Hasklig}
\defaultfontfeatures{Mapping=tex-text,Scale=MatchLowercase,Ligatures=TeX}

\usepackage{pgfplots}
\usepackage{pgfplotstable}
\usepackage{bchart}

\title{Data Structures \& Algorithms}
\author{Ricard Solé Casas}
\providecommand{\institute}[1]{}
\institute{Google UK \and Ada National College for Digital Skills}
\date{\today}

\begin{document}
\maketitle

\vspace*{\fill}

\section*{Foreword}

Some of the code samples in this hand-in are in Haskell, as opposed to
pseudocode. This is because it made for easier testing and data modeling
for the problems. It happens to be a very expressive language that
closely resembles mathematical functions.

\section*{Declaration}

I confirm that the submitted coursework is my own work and that all
material attributed to others (whether published or unpublished) has
been clearly identified and fully acknowledged and referred to original
sources. I agree that the College has the right to submit my work to the
plagiarism detection service. TurnitinUK for originality checks.

\section*{Acknowledgements}

I'd like to thank my partner Shannon for her continued support and
challenges that help me grow, both professionally and personally. I
would also like to thank all of you who also helped me get here.

\vspace*{\fill}

{
\setcounter{tocdepth}{2}
\tableofcontents
}
\chapter{Analysis of Sorting
Algorithms}\label{analysis-of-sorting-algorithms}

\section{Sort time growth}\label{sort-time-growth}

\pgfplotstableread[col sep=semicolon,trim cells,row sep=crcr]{
  interval; sortB     ; sortS        ; sortI        ; sortM \\
  10000   ; 194       ; 49           ; 31           ; 4     \\
  50000   ; 6318      ; 1165         ; 481          ; 20    \\
  100000  ; 24193     ; 4472         ; 1742         ; 33    \\
  150000  ; 54464     ; 10321        ; 4102         ; 40    \\
  200000  ; 96889     ; 18180        ; 7429         ; 76    \\
}\sortdata

\begin{tikzpicture}
  \begin{axis}[
    title={Sorting algorithms runtime growth},
    width=\textwidth,
    height=.8\textwidth,
    scaled ticks=false, tick label style={/pgf/number format/fixed},
    xlabel={Number of items},
    ylabel={Time in ms (less is better)},
    xmin=0, xmax=200000,
    ymin=0, ymax=100000,
    xtick=data,
    ytick=data,
    legend pos=north west,
    ymajorgrids=true,
    grid style=dashed,
    every axis plot/.append style={ultra thick},
  ]
  \addplot+[red] table[x=interval,y=sortB]{\sortdata};
  \addplot+[orange] table[x=interval,y=sortS]{\sortdata};
  \addplot+[yellow] table[x=interval,y=sortI]{\sortdata};
  \addplot+[green] table[x=interval,y=sortM]{\sortdata};
  \legend{Bubble Sort,Selection Sort,Insertion Sort,Merge Sort}
  \end{axis}
\end{tikzpicture}

\section{Runtime Comparison}\label{runtime-comparison}

\pgfplotstableread[col sep=semicolon,trim cells,row sep=crcr]{
  algorithm      ; time  \\
  Bubble Sort    ; 96889 \\
  Selection Sort ; 18180 \\
  Insertion Sort ; 7429  \\
  Merge Sort     ; 76    \\
}\runtimedata

\begin{tikzpicture}
  \begin{axis}[
    ybar,
    bar width=.5cm,
    title={Runtime in ms per algorithm per \# of items},
    width=\textwidth,
    height=.8\textwidth,
    xlabel={Number of items},
    ylabel={Time in ms (less is better)},
    scaled ticks=false, tick label style={/pgf/number format/fixed},
    ytick=data,
    xtick=data,
    legend pos=north west,
    ymajorgrids=true,
    grid style=dashed,
  ]
  \addplot+[red] table[x=interval,y=sortB]{\sortdata};
  \addplot+[orange] table[x=interval,y=sortS]{\sortdata};
  \addplot+[yellow] table[x=interval,y=sortI]{\sortdata};
  \addplot+[green] table[x=interval,y=sortM]{\sortdata};
  \legend{Bubble Sort,Selection Sort,Insertion Sort,Merge Sort}
  \end{axis}
\end{tikzpicture}

\section[Reflection]{\texorpdfstring{Reflection\footnote{Selection sort
  and benchmark utility code can be found Appendix A.}}{Reflection}}\label{reflectionlongnote}

We know that the complexity for the algorithms benchmarked above
---\(O(n^2)\), \(O(n^2)\), \(O(n^2)\) and \(O(n\log_2(n))\)
respectively---, and we can see that mapped accurately in the plots
above.

The first plot, the graph displayed on section 1.1, shows the increment
in time for the \emph{bubble}, \emph{selection}, and \emph{insertion}
sort algorithms as exponential. Their curves vary greatly because
\texttt{Big\ O} is not meant to give an accurate description of the
time, just of how it'll scale. The point being, they all scale in a
similar manner. The green line represents the \emph{merge} sort
algorithm which is, along with \emph{quicksort}, considered to be the
best algorithm overall. We can barely see it scale because the plot is
drawn next to the quadratic plot of \emph{bubble} sort. If we look
closely, though, we can see it almost resembles a linear time
complexity, but not quite.

The second plot, a bar graph displayed on section 1.2, uses the same
colors and the same data to represent, perhaps more clearly, how each
algorithm fairs amongst each other using the same dataset.

\chapter{Segregate Even and Odd
numbers}\label{segregate-even-and-odd-numbers}

\section{Task}\label{task}

Given an array \texttt{A{[}{]}}, write an algorithm in pseudocode that
segregates even and odd numbers. The algorithm should put all even
numbers first, and then odd numbers.

\section[Implementation]{\texorpdfstring{Implementation\footnote{Full
  program sample can be found on Appendix B}}{Implementation}}\label{implementation2}

\inputminted[firstline=9,lastline=17]{haskell}{code_samples/Isolate.hs}

\section{Complexity analysis}\label{complexity-analysis}

For the algorithm implemented in section 2.2:

\begin{itemize}
\tightlist
\item
  \texttt{filter}ing through an array for \texttt{even} or \texttt{odd}
  numbers has a complexity of \(O(n)\), where \(n\) is the length of the
  given list.
\item
  \texttt{(++)} has a complexity of \(O(m)\)\footnote{https://goo.gl/yZu7Ig},
  where \(m\) is the length of the second list.

  \begin{itemize}
  \tightlist
  \item
    \(m = length(odds(xs))\) in our case
  \end{itemize}
\item
  With this we can conclude that the complexity of this algorithm is
  \(O(2n+m)\), which can be simplified to \(O(n+m)\).
\end{itemize}

\chapter[Recursion]{\texorpdfstring{Recursion\footnote{Full code samples
  can be found on Appendix C}}{Recursion}}\label{recursion4}

\section{Pizza cutting}\label{pizza-cutting}

\subsection{Problem definition}\label{problem-definition}

When you cut a pizza, you cut along a diameter of the pizza. Let
\texttt{pizza(n)} be the number of slices of pizza that exist after you
have made \texttt{n} cuts, where \texttt{n\ ≥\ 1}. For example,
\texttt{pizza(2)\ =\ 4} because there are four slices after two diagonal
cuts.

Write a recursive method \texttt{pizza(n)} to return the number of
slices and verify the correctness of your method when the pizza is cut 4
times \emph{(3 points)}.

\subsection{Implementation}\label{implementation}

\inputminted[firstline=6,lastline=9]{haskell}{code_samples/Pizza.hs}

\section{Parking}\label{parking}

\subsection{Problem definition}\label{problem-definition-1}

A bunch of motorcycles and cars want to parallel park on a street. The
street can fit \texttt{n} motorcycles, but one car take up three
motorcycle spaces. Let \texttt{a(n)} be the number of arrangements of
cars and motorcycles on a street that fits n motorcycles \emph{(7
points)}.

\subsection{Implementation}\label{implementation-1}

\inputminted[firstline=6,lastline=11]{haskell}{code_samples/Parking.hs}

\subsection{\texorpdfstring{Possible Arrangements
\(n = 6\)}{Possible Arrangements n = 6}}\label{possible-arrangements-n-6}

MMMC, MMCM, MCMM, CMMM, MMMMMM, CC.

\chapter{Appendix A}\label{appendix-a}

\section{Benchmark.java}\label{benchmark.java}

\inputminted{java}{code_samples/Benchmark.java}

\section{SelectionSort.java}\label{selectionsort.java}

\inputminted{java}{code_samples/SelectionSort.java}

\chapter{Appendix B}\label{appendix-b}

\section{Isolate.hs}\label{isolate.hs}

\inputminted{haskell}{code_samples/Isolate.hs}

\chapter{Appendix C}\label{appendix-c}

\section{Pizza.hs}\label{pizza.hs}

\inputminted{haskell}{code_samples/Pizza.hs}

\section{Parking.hs}\label{parking.hs}

\inputminted{haskell}{code_samples/Parking.hs}

\end{document}
